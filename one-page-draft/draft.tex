\documentclass[12pt]{report}
\usepackage[utf8]{inputenc}
\usepackage[left=2.5cm,right=2.5cm,top=2cm,bottom=2cm]{geometry}
\usepackage{xcolor}
\usepackage{hyperref}
\author{Andrew Buhagiar}
\title{One Page Draft - JavaScript Framework for Actor-Based Programming}
\definecolor{dkgreen}{rgb}{0,0.6,0}
\definecolor{gray}{rgb}{0.5,0.5,0.5}
\definecolor{mauve}{rgb}{0.58,0,0.82}
\setlength{\parindent}{0pt}
\setlength{\parskip}{7pt}
\begin{document}
\maketitle
The project aims to engineer a JavaScript framework for building actor-based systems which will use the web as its global distributed platform. The performance and usability of the artifact will then be analysed through empirical measurement.

Actors are concurrent units of computation which may perform three operations in response to incoming messages. They may send messages to other actors, decide how to handle the next message and create a finite set of new actors. Actors maintain an isolated state which may be changed in response to incoming messages\cite{agha_1985}. Governed by their limited behaviour, actors communicate with each other to constitute a distributed system. Actors are ideal for building message driven reactive systems\cite{reactivemanifesto} which are scalable and fault tolerant. This model powers the telecommunications industry and is more recently used for implementing distributed systems.

The framework will provide operations which will enable the programmer to easily set up actors and the communication between them. Such operations will allow the programmer to create, terminate, and send messages to actors. Internally the framework will handle the distributed communication between actors as well as the isolation of actor states. Such actors may be running locally or remotely on other devices. The framework will allow the programmer to use WebSockets to pass messages through the web to communicate with the remote actors. The mode of communication will be tailored depending on the location of the two communicating actors to optimise the speed of message delivery. The framework will also handle the blocking and scheduling of actors during runtime, determining which actors will run at any point in time. Measures will be taken to ensure fairness between the running concurrent actors.

Having implemented necessary features for the actor model framework, the focus may shift to explore more advanced features. For instance, the framework may allow actors to freeze and persist. The programmer could then respawn or replicate the persisted actor with the respective state and mailbox. The framework may also provide location transparency. This will allow programmers to treat actors in foreign devices in the same way as actors which are running locally, thus simplifying logic on distributed networks. It may be the case that a cluster of nodes will be spawned to perform local parallel computing. This will take advantage of the multiple cores that are usually present in computers and servers.

Under varying computational load, the response times of the actors will be measured and analysed to assess the system’s scalability. The memory, time and generated network traffic will also be analysed to measure the footprint the framework has on the devices it is running on. Experiments can be carried out in different environments, such as locally and over the web. The experiments will determine the suitability of the actor model system for building distributed systems.
\bibliographystyle{ieeetr}
\bibliography{references}
\end{document}