\documentclass[12pt]{report}
\usepackage[utf8]{inputenc}
\usepackage[left=2cm,right=2cm,top=2cm,bottom=2cm]{geometry}
\usepackage{xcolor}
\usepackage{hyperref}
\author{Andrew Buhagiar}
\title{One Page Draft - JavaScript Framework for Actor-Based Programming}
\definecolor{dkgreen}{rgb}{0,0.6,0}
\definecolor{gray}{rgb}{0.5,0.5,0.5}
\definecolor{mauve}{rgb}{0.58,0,0.82}
\setlength{\parindent}{0pt}
\setlength{\parskip}{7pt}
\begin{document}
\maketitle
The project aims to provide a JavaScript framework for designing systems using the actor model. This will allow developers to simplify code that will be running concurrently in separate processes to build distributed parallel systems.

Actors are units of computation which may perform three operations in response to incoming messages\cite{agha_1986}. They may send messages to other actors, decide how to handle the next message and create a finite set of new actors. Actors maintain a state which may be changed in response to incoming messages. Governed by their limited behaviour, isolated actors communicate with each other to build a distributed system.

The framework will provide functions which will enable the programmer to easily set up actors and communication between them. Such functions will allow the programmer to create, terminate, and send messages to actors. Alternate functions will be provided to accommodate for more specific behaviour, such as deciding whether the actor has a state, or whether sending a message will be blocking. Internally the framework will handle communication between actors as well as the isolation of actor states, encapsulating actor logic which is irrelevant to the programmer. 

This thesis will use the developed basic actor model framework to explore more advanced features. The actors should be able to freeze and persist. This implies that actors can be frozen at any time through a function call and its state including its mailbox can be persisted on a file. The programmer can then respawn or replicate the actor with the mailbox it had previously and resume operation. The thesis will also focus on location transparency. The programmer will be able to provide an address scheme of other running nodes which host a set of actors. These nodes may be running locally or remotely on other computers. The programmer can reason about actor systems without having to worry on where the actors are deployed. This allows exploits for not only multicore systems but for networks with multiple devices.

The system’s quality will be determined by several factors inspired by the reactive manifesto\cite{reactivemanifesto}. Actors will be responsive such that they will process and act on received messages fast and consistently. The framework must also minimise the degradation of performance in the event of having a considerable number of spawned actors. Computer resources must be distributed fairly throughout the unblocked actors. The framework must also be message driven and asynchronous, ensuring that no actors rely on a global time to execute events. Finally, resilience is offered by providing means of handling failing actors. The programmer will be able to specify the procedure for a crashing actor, as well as define a hierarchical model to ensure that actors are supervised by other actors to increase reliability.
\bibliographystyle{ieeetr}
\bibliography{references}
\end{document}