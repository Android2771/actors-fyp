\documentclass[12pt]{report}
\usepackage[left=2.5cm,right=2.5cm,top=2cm,bottom=2cm]{geometry}
\usepackage[utf8]{inputenc}
\usepackage{xcolor}
\usepackage{hyperref}
\usepackage{listings}
\usepackage{float}
\author{Andrew Buhagiar}
\title{Progress Report - JavaScript Framework for Actor-Based Programming}
\definecolor{dkgreen}{rgb}{0,0.6,0}
\definecolor{gray}{rgb}{0.5,0.5,0.5}
\definecolor{mauve}{rgb}{0.58,0,0.82}
\renewcommand*\thesection{\arabic{section}}
\setlength{\parindent}{0pt}
\setlength{\parskip}{7pt}

\lstdefinelanguage{JavaScript}{
  keywords={break, case, catch, continue, debugger, default, delete, do, else, false, finally, for, function, if, in, instanceof, new, null, return, switch, this, throw, true, try, typeof, let, const, void, while, with, class, export, boolean, throw, implements, import, this},
  morecomment=[l]{//},
  morecomment=[s]{/*}{*/},
  morestring=[b]',
  morestring=[b]",
  ndkeywords={},
  keywordstyle=\color{blue}\bfseries,
  ndkeywordstyle=\color{darkgray}\bfseries,
  identifierstyle=\color{black},
  commentstyle=\color{purple}\ttfamily,
  stringstyle=\color{red}\ttfamily,
  sensitive=true
}

\lstset{frame=tb,
  language=JavaScript,
  aboveskip=3mm,
  belowskip=3mm,
  showstringspaces=false,
  columns=flexible,
  basicstyle={\small\ttfamily},
  numbers=none,
  numberstyle=\tiny\color{gray},
  keywordstyle=\color{blue},
  commentstyle=\color{dkgreen},
  stringstyle=\color{mauve},
  breaklines=true,
  breakatwhitespace=true,
  tabsize=3,
  numbers=left
}

\begin{document}
\maketitle
\tableofcontents
\newpage
%page for each section
\section{Abstract}
JavaScript is used in most modern browsers as well as standalone applications using Node.js. Despite JavaScript being the language of the web, it lacks an intuitive way to program in a concurrent and distributed fashion. This dissertation aims to engineer a JavaScript framework for building actor-based systems which will use the web as its global distributed platform, and explore the performance and scalability of applications built using this framework.
\section{Introduction and Motivation}
Actors\cite{43years} are concurrent units of computation proposed by \cite{hewitt1973session}. They communicate with each other using messages which may be stored on the receiving actor's message queue. Actors are event driven as they execute a certain behaviour when processing the next message in the queue. This behaviour may involve sending messages to other actors, changing the actor's behaviour as well as creating a finite set of new actors~\cite{hewitt2010actor}. Actors communicate with each other to constitute a distributed system. Actors are ideal for building message driven reactive systems~\cite{reactivemanifesto} which are scalable and fault tolerant. This model powers the telecommunications industry and is more recently used for implementing distributed systems in languages such as Scala Actors and Akka~\cite{haller2012integration}.

By providing a JavaScript framework which allows the programmer to reason about code through the construction and message passing of actors, it will become easier to distribute workload amongst different threads or systems. This workload will be computed concurrently by actors with an isolated state. Assuming that the actor system executes the behaviour to process each message sequentially in an isolated manner, the programmer does not need to worry about low-level race conditions~\cite{43years}\cite{haller2010isolated}. 

By using the actor model, the programmer may also choose to mitigate JavaScript's prevalent "callback hell" by opting to use actors in favour of callback functions, resulting in cleaner code. Due to their lightweight nature, actors may also be used to divide responsibilities for an application, such as using actors to represent microservices or serve as web endpoints\cite{hewitt2010actor}. Using the actor model provides fault tolerance if mechanisms are in place to ensure that the failure of actors are contained to just the failing actor. Such an actor can then be respawned by other overseeing actors.

This dissertation aims to create one or more actor model prototypes which will be used to explore the feasibility of using the actor model in both a quantitative and qualitative manner. The performance and scalability of systems for an increasing number of actors will be explored empirically. The development cost of using the actor model in favour of plain JavaScript will also be explored in a qualitative manner.
\section{Background Research and Literature Review}
\subsection{The Importance of Concurrency on JavaScript}
While it can be argued that JavaScript is essential to modern browsers, there is also a growing popularity for server-side JavaScript. Some tasks may require significant computational power to execute. By writing concurrent code, a developer is able to write applications that utilise multiple threads, cores, or devices. If some tasks require input from the user or other tasks, resources can be allocated to the working tasks instead. JavaScript and Node.js do not rely on multithreading to support concurrent execution of such tasks~\cite{highperformance}, which is an issue when high performance computing is required. However, Web Workers can be used to bring parallel computation to the language and uses a similar philosophy to that of the actor model~\cite{concurrencyjs}.
\subsection{The History of the Actor Model}
The Actor Model was first presented as a concurrency model in 1973 by Carl Hewitt~\cite{hewitt1973session}. Actors were refined by Gul A. Agha in 1985~\cite{agha1985actors} as computational agents which map incoming communication to a certain behaviour. This behaviour may be commuincating with other actors, decide how to handle the next message and creating new actors. Since then, the Actor model has been integrated into mainstream programming languages such as Akka, Scala and Erlang~\cite{43years}\cite{haller2012integration}. The success of the Actor model may be attributed to its message driven and isolated nature which allows for easier concurrent programming, and their resilience against failure as actors are isolated and can be easily replicated by other actors\cite{reactivemanifesto}.
\subsection{Attempts to bring the Actor Model onto JavaScript}
Thanks to user contributions, a number of NPM packages are available to bring the actor model onto JavaScript (see \url{https://www.npmjs.com/search?q=keywords:actors}). A notable project called Akka.js\cite{stivan2015akka} allows developers to cross-compile Akka to JavaScript browsers and server-side runtimes. This framework allows developers to build distributed applications through separate browsers using WebRTC. It also incentivises developers to only have to be knowledgeable with Scala to deploy applications on both the browser and the server.
\section{Aims and Objectives}
The framework will provide operations which will enable the programmer to easily set up actors and the communication between them. Such operations will allow the programmer to create, terminate, and send messages to actors. Internally the framework will handle the distributed communication between actors as well as the isolation of actor states. Such actors may be running locally or remotely on other devices. The framework will allow the programmer to use a facility such as WebSockets to pass messages through the web to communicate with the remote actors. The mode of communication will be tailored depending on the location of the two communicating actors to optimise the speed of message delivery. The framework will also handle the blocking and scheduling of actors during runtime, determining which actors will run at any point in time. Measures will be taken to ensure fairness between the running concurrent actors.

Having implemented necessary features for the actor model framework, the focus may shift to explore more advanced features. For instance, the framework may allow actors to freeze and persist. The programmer could then respawn or replicate the persisted actor with the respective state and mailbox. The framework may also provide location transparency. This will allow programmers to treat actors in foreign devices in the same way as actors which are running locally, thus simplifying logic on distributed networks. It may be the case that a cluster of nodes will be spawned to perform local parallel computing. This will take advantage of the multiple cores that are usually present in computers and servers.

\section{Proposed Solution}
The figure below shows a basic prototype of an actor model framework implementation for a single Node.js application.
\begin{figure}[H]
  \begin{lstlisting}[gobble=2]
    const EventEmitter = require('events');
    class MessageEmitter extends EventEmitter {}
    const messageEmitter = new MessageEmitter();
    let i = 0;
    module.exports = {
        spawn: (state, behaviour) => {
            const name = (i++).toString();
            const actor = { name, state, mailbox: [] };
            messageEmitter.on(name, () => {
                let message = actor.mailbox.shift();
                if (message !== undefined)
                    behaviour(actor.state, message);
            });
            return actor;
        },
        terminate: (actor) => {
            messageEmitter.removeAllListeners(actor.name);
        },
        send: (actor, message) => {
            actor.mailbox.push(message);
            messageEmitter.emit(actor.name);
        }
    };
  \end{lstlisting}
  \caption{Naive Single Thread Actor System}
\end{figure}
The implementation exploits the Node.js event emitter system~\cite{nodeevents} to use it for scheduling which actors will run their behaviour at a time.

This example exposes the fundamental functions to spawn actors and send messages to actors. Inspired by Akka~\cite{stivan2015akka}, when spawning an actor a reference to it is returned which can then be used to send messages to this actor. Sending messages involves pushing onto the actor's mailbox and emitting the event that a message has arrived for the actor. It is then up to Node.js to schedule when the event is handled and the message is processed through the defined behaviour closure.

This implementation can be extended to provide for concurrent and distributed systems over different Node.js applications or WebWorkers running on potentially distributed systems. The solution is not trivial as communication links must be established between actors on a distributed platform and adaptive implementations might be required for different environments. Other features to explore and test is covered in the next section
\section{Evaluation Plan}
% youre doing a scalability test for n actors, this is how you evaluate
% overhead vs actual computation internally, one actor o
%mention case study of actor system and benchmarking (graphs)
%Gantt chart - implementation
Several experiments will be carried out to empirically measure the response times of the actors under varying computational load to assess the system’s scalability. The time and space complexity and generated network traffic will also be analysed to measure the footprint the framework has on the devices it is running on. Moreover, a case study will be done to perform a qualitative comparison between JavaScript code using the actor model to plain JavaScript. The experiments will add to the existing evidence of the practicality of implementing the actor model framework in JavaScript. The experiments will also assess the system's performance and scalability through quantitative analysis. Finally, through the qualitative case study the usability of the actor model will be shown when using JavaScript on the web.
\section{Deliverables}
\bibliographystyle{ieeetr}
\bibliography{references}
\end{document}